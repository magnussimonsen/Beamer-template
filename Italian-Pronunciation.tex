\section{Unit 1: Italian Pronunciation}

% --- Slide 1: Overview ---
\blueheader
\begin{frame}{Italian Pronunciation — Overview}
\begin{blue*}{Goal}
Get a quick, practical map of Italian sounds without IPA:
\begin{itemize}
  \item \textbf{Vowels}: stable, pure sounds (no diphthongs like in English).
  \item \textbf{C/G rules}: different sounds before \textit{e, i} and with \textit{h}.
  \item \textbf{Special combos}: \textbf{gli}, \textbf{gn}, \textbf{sc(e/i)}, \textbf{z}, \textbf{s}.
  \item \textbf{Stress \& doubles}: where to put emphasis; double consonants matter!
\end{itemize}
\end{blue*}

\begin{green*}{Tip}
Italian spelling is highly phonetic. Learn the few core rules and you can read almost anything.
\end{green*}
\end{frame}

% --- Slide 2: Vowels ---
\blueheader
\begin{frame}{Vowels (pure and stable)}
\begin{red*}{Rule}
Italian has five written vowels with \emph{stable} values (avoid gliding):
\begin{itemize}
  \item \textbf{a} as in “father”: \textit{pasta, casa}
  \item \textbf{e} either like “they” (closed) or “bet” (open): \textit{pera}, \textit{caffè}
  \item \textbf{i} as in “machine”: \textit{vino}, \textit{Italia}
  \item \textbf{o} either like “go” (closed) or “off” (open): \textit{porta}, \textit{però}
  \item \textbf{u} as in “flute”: \textit{luna}, \textit{tutti}
\end{itemize}
\end{red*}

\begin{green*}{Examples}
\textbf{pane} (bread), \textbf{sera} (evening), \textbf{vino} (wine), \textbf{sole} (sun), \textbf{luna} (moon).
\end{green*}
\end{frame}

% --- Slide 3: C/G rules + H ---
\blueheader
\begin{frame}{C/G before E/I and the role of H}
\begin{red*}{Rule (C)}
\begin{itemize}
  \item \textbf{c + a/o/u} = hard “k”: \textit{casa}, \textit{cosa}, \textit{cuore}
  \item \textbf{c + e/i} = soft “ch” (like “church”): \textit{cena}, \textit{cinema}
  \item \textbf{ch + e/i} = hard “k”: \textit{che}, \textit{chi} (as in \textit{spaghetti})
\end{itemize}
\end{red*}

\begin{red*}{Rule (G)}
\begin{itemize}
  \item \textbf{g + a/o/u} = hard “g” (as in “go”): \textit{gatto}, \textit{gola}, \textit{gustare}
  \item \textbf{g + e/i} = soft “j” (as in “jam”): \textit{gelato}, \textit{gioco}
  \item \textbf{gh + e/i} = hard “g”: \textit{ghetto}, \textit{ghiaccio}
\end{itemize}
\end{red*}

\begin{green*}{Examples}
\textbf{cena} vs \textbf{che}; \textbf{giro} vs \textbf{ghiro}; \textbf{spaghetti}, \textbf{cinema}, \textbf{gelato}.
\end{green*}
\end{frame}

% --- Slide 4: Special letter groups ---
\blueheader
\begin{frame}{Special groups: \textit{gli}, \textit{gn}, \textit{sc(e/i)}, \textit{z}, \textit{s}}
\begin{red*}{Rules}
\begin{itemize}
  \item \textbf{gli} = single sound (like “lli” merged): \textit{famiglia}, \textit{figlio}
  \item \textbf{gn} = “ny” (like Spanish ñ): \textit{gnocchi}, \textit{bagno}
  \item \textbf{sc + e/i} = “sh”: \textit{scena}, \textit{scimmia}; \ \textbf{sca/sco/scu} = “sk”: \textit{scala}, \textit{scuola}
  \item \textbf{z} = either “ts” or “dz” (both common): \textit{pizza} (“ts”), \textit{zucchero} (“dz” in many regions)
  \item \textbf{s} = “s” or “z” sound; between vowels often “z” (voiced): \textit{casa} (often “z”-like)
\end{itemize}
\end{red*}

\begin{green*}{Examples}
\textbf{famiglia}, \textbf{gnocchi}, \textbf{scena}, \textbf{scuola}, \textbf{pizza}, \textbf{zucchero}, \textbf{casa}.
\end{green*}
\end{frame}

% --- Slide 5: Stress and Double Consonants ---
\blueheader
\begin{frame}{Stress \& Double consonants}
\begin{red*}{Rule}
\begin{itemize}
  \item Stress is usually on the \textbf{second-to-last} syllable: \textit{ta-VO-lo} (table).
  \item Words with a final accent mark stress the \textbf{last} syllable: \textit{citt\`a}, \textit{perch\'e}, \textit{caff\`e}.
  \item \textbf{Double consonants} are \emph{held longer} and change meaning: 
    \textit{casa} (house) vs \textit{cassa} (cash register), 
    \textit{pala} (shovel) vs \textit{palla} (ball).
\end{itemize}
\end{red*}

\begin{cyan*}{Practice}
Clap or tap on the stressed syllable: \textit{a-MI-co}, \textit{Do-me-ni-ca}, \textit{Ma-cca-ri-ni}, \textit{caff\`e}.
\end{cyan*}
\end{frame}

% --- Slide 6: Exercise ---
\cyanheader
\begin{frame}{Exercise: Read aloud and mark the stress}
\begin{cyan*}{Task}
Mark the stressed syllable and identify any special sounds:
\begin{enumerate}
  \item \textbf{gelato}, \textbf{gioco}, \textbf{ghiaccio}
  \item \textbf{cena}, \textbf{che}, \textbf{cinema}
  \item \textbf{famiglia}, \textbf{gnocchi}, \textbf{bagno}
  \item \textbf{scena}, \textbf{scimmia}, \textbf{scuola}
  \item \textbf{citt\`a}, \textbf{perch\'e}, \textbf{caff\`e}
  \item \textbf{casa} vs \textbf{cassa}, \textbf{pala} vs \textbf{palla}
\end{enumerate}
\end{cyan*}
\end{frame}
