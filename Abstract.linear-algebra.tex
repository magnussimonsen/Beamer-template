% --- Slide 1: Vector Space Definition ---
\blueheader
\begin{frame}{Abstract Linear Algebra: Vector Space}
\begin{blue*}{Definition}
A \textbf{vector space} over a field $\mathbb{F}$ is a set $V$ equipped with
\begin{itemize}
    \item vector addition $+: V \times V \to V$,
    \item scalar multiplication $\cdot: \mathbb{F} \times V \to V$,
\end{itemize}
satisfying for all $u,v,w \in V$, $\alpha,\beta \in \mathbb{F}$:
\begin{enumerate}
    \item $u+v = v+u$ \hfill (commutativity)
    \item $(u+v)+w = u+(v+w)$ \hfill (associativity)
    \item There exists $0 \in V$ such that $v+0=v$ \hfill (additive identity)
    \item For each $v$, there exists $-v$ with $v+(-v)=0$ \hfill (additive inverse)
    \item $\alpha(u+v) = \alpha u + \alpha v$ \hfill (distributivity I)
    \item $(\alpha+\beta) v = \alpha v + \beta v$ \hfill (distributivity II)
    \item $(\alpha\beta)v = \alpha(\beta v)$ \hfill (associativity of scalar mult.)
    \item $1 \cdot v = v$ \hfill (unit property)
\end{enumerate}
\end{blue*}
\end{frame}

% --- Slide 2: Remark ---
\redheader
\begin{frame}{Abstract Linear Algebra: Remark}
\begin{red*}{Remark}
A vector space by itself has only the structure given by addition and scalar multiplication.  
\medskip

\begin{itemize}
    \item It does \emph{not} have a notion of \textbf{length}, \textbf{angle}, or \textbf{orthogonality}.
    \item These geometric notions are \textbf{induced} when we add an \textbf{inner product}, and from it, a norm.
\end{itemize}
\end{red*}
\end{frame}

% --- Slide 3: Inner Product ---
\blueheader
\begin{frame}{Abstract Linear Algebra: Inner Product}
\begin{blue*}{Definition}
An \textbf{inner product} on a real vector space $V$ is a function
\[
\langle \cdot, \cdot \rangle : V \times V \to \mathbb{R}
\]
such that for all $u,v,w \in V$, $\alpha \in \mathbb{R}$:
\begin{enumerate}
    \item $\langle u,v \rangle = \langle v,u \rangle$ \hfill (symmetry)
    \item $\langle u+v,w \rangle = \langle u,w \rangle + \langle v,w \rangle$ \hfill (linearity)
    \item $\langle \alpha u, v \rangle = \alpha \langle u,v \rangle$ \hfill (homogeneity)
    \item $\langle v,v \rangle \ge 0$, and $\langle v,v \rangle = 0 \iff v=0$ \hfill (positivity)
\end{enumerate}
\end{blue*}
\end{frame}

% --- Slide 4: Norm ---
\redheader
\begin{frame}{Abstract Linear Algebra: Norm}
\begin{red*}{Definition}
Given an inner product $\langle \cdot, \cdot \rangle$ on $V$, the induced \textbf{norm} is defined by
\[
\|v\| = \sqrt{\langle v, v \rangle}, \quad v \in V.
\]

This norm introduces the notions of:
\begin{itemize}
    \item \textbf{Length}: $\|v\|$
    \item \textbf{Angle}: $\cos \theta = \dfrac{\langle u, v \rangle}{\|u\|\|v\|}$
    \item \textbf{Orthogonality}: $u \perp v \iff \langle u, v \rangle = 0$
\end{itemize}
\end{red*}
\end{frame}

% --- Slide 5: Example ---
\greenheader
\begin{frame}{Abstract Linear Algebra: Example}
\begin{green*}{Example}
Let $V = \mathbb{R}^2$ with the standard inner product
\[
\langle u, v \rangle = u_1 v_1 + u_2 v_2.
\]

For $u = (3,4)$ and $v = (4,-3)$:
\[
\langle u, v \rangle = 3 \cdot 4 + 4 \cdot (-3) = 12 - 12 = 0.
\]

Thus $u \perp v$.  
Also $\|u\| = \sqrt{3^2+4^2}=5$.
\end{green*}
\end{frame}

% --- Slide 6: Exercise ---
\cyanheader
\begin{frame}{Abstract Linear Algebra: Exercise}
\begin{cyan*}{Exercise}
Let $u=(1,2,2)$ and $v=(2,0,1)$ in $\mathbb{R}^3$ with the standard inner product.
\begin{enumerate}
    \item Compute $\langle u,v \rangle$.
    \item Find $\|u\|$ and $\|v\|$.
    \item Determine the cosine of the angle between $u$ and $v$.
\end{enumerate}
\end{cyan*}
\end{frame}
