
% ======================================================================
% Beamer template for lectures (16:9 handout version)
% MIT © 2025 Magnus Simonsen
%
% This template provides:
%   - Clean 16:9 Beamer slides optimized for handouts.
%   - Color macros and tcolorboxes for structure and emphasis.
%   - Optional 2-up or 4-up handout printing layouts.
% ======================================================================

\documentclass[11pt, aspectratio=169, handout]{beamer}

% -------------------------------
% Encoding, language, typography
% -------------------------------
\usepackage[utf8]{inputenc}
\usepackage[T1]{fontenc}
\usepackage[english]{babel}
\usepackage{lmodern}
\usepackage{microtype}

% -------------------------------
% Beamer theme and navigation
% -------------------------------
\usetheme{Copenhagen}
\setbeamercovered{transparent}
\setbeamertemplate{headline}{}                % Remove header line
\setbeamertemplate{footline}[frame number]    % Frame number only
\setbeamertemplate{navigation symbols}{}      % Hide navigation icons

% -------------------------------
% Commonly used packages
% -------------------------------
\usepackage{amsmath, amsfonts, amsthm, amssymb}
\usepackage{graphicx}
\usepackage{xcolor}
\usepackage{hyperref}
\usepackage{siunitx}
\sisetup{detect-all, per-mode=symbol, separate-uncertainty=true}
\usepackage{pgfplots}
\pgfplotsset{compat=1.18}
\usepackage{tikz}
\usepackage[most]{tcolorbox}
\usepackage{fontawesome5}
\usepackage{minted} % requires --shell-escape for compilation
\usepackage{tipa}

% Optional: multi-slide handout layout
\usepackage{pgfpages}
% \pgfpagesuselayout{2 on 1}[a4paper, border shrink=5mm]
% \pgfpagesuselayout{4 on 1}[a4paper, border shrink=5mm]

% -------------------------------
% Hyperlink settings
% -------------------------------
\hypersetup{
  pdftitle={R2: Lecture Notes},
  pdfauthor={Magnus Simonsen},
  colorlinks=true,
  linkcolor=blue,
  urlcolor=cyan,
  citecolor=magenta
}

% -------------------------------
% Custom colors
% -------------------------------
\colorlet{myblue}{black!35!blue}
\colorlet{myred}{black!35!red}
\colorlet{mygreen}{black!35!green}
\colorlet{mycyan}{cyan}
\definecolor{mymagenta}{rgb}{0.54,0.0,0.54}

% -------------------------------
% Frame color shortcuts
% -------------------------------
\newcommand{\blueframe}{%
  \setbeamercolor{frametitle}{bg=myblue, fg=white}%
}
\newcommand{\redframe}{%
  \setbeamercolor{frametitle}{bg=myred, fg=white}%
}
\newcommand{\greenframe}{%
  \setbeamercolor{frametitle}{bg=mygreen, fg=white}%
}
\newcommand{\cyanframe}{%
  \setbeamercolor{frametitle}{bg=mycyan, fg=black}%
}
\newcommand{\magentaframe}{%
  \setbeamercolor{frametitle}{bg=mymagenta, fg=white}%
}

% -------------------------------
% Warning and info macros
% -------------------------------
\newcommand{\warn}{%
  \textcolor{orange}{\faExclamationTriangle\ \textbf{Not part of the syllabus}}%
}
\newcommand{\warnCustomMsg}[2][orange]{%
  \textcolor{#1}{\faExclamationTriangle\ \textbf{#2}}%
}
\newcommand{\info}{%
  \textcolor{blue}{\faInfoCircle\ \textbf{For your information}}%
}
\newcommand{\infoCustomMsg}[2][myblue]{%
  \textcolor{#1}{\faInfoCircle\ \textbf{#2}}%
}

% -------------------------------
% tcolorbox environments (color-based names)
% Usage: \begin{bluebox}{Title}...\end{bluebox}
% -------------------------------
\tcbset{
  colback=white,
  fonttitle=\bfseries,
  breakable,
  enhanced,
  top=4pt,bottom=4pt,left=6pt,right=6pt,
  boxsep=0pt
}

\newtcbtheorem[number within=section]{bluebox}{Definition}%
{colframe=myblue}{blue}

\newtcbtheorem[use counter from=bluebox]{redbox}{Theorem}%
{colframe=myred}{red}

\newtcbtheorem[use counter from=bluebox]{greenbox}{Example}%
{colframe=mygreen}{green}

\newtcbtheorem[use counter from=bluebox]{cyanbox}{Exercise}%
{colframe=mycyan}{cyan}

\newtcbtheorem[use counter from=bluebox]{magentabox}{Explore}%
{colframe=mymagenta}{magenta}

% Optionally number theorems on slides:
% \setbeamertemplate{theorems}[numbered]

% ======================================================================
% Document
% ======================================================================
\begin{document}

% Example slide
% \blueframe
% \begin{frame}
%  \frametitle{Linear Transformations}
%  \begin{bluebox}{Linear Transformation}
%    A mapping $T: V \to W$ is called
%    \emph{linear} if it satisfies
%    \[
%      T(u+v) = T(u) + T(v), \qquad T(\alpha v) = \alpha T(v).
%       \]
%  \end{bluebox}
% \end{frame}

% Content modules
% % --- Slide 1: Vector Space Definition ---
\blueheader
\begin{frame}{Abstract Linear Algebra: Vector Space}
\begin{blue*}{Definition}
A \textbf{vector space} over a field $\mathbb{F}$ is a set $V$ equipped with
\begin{itemize}
    \item vector addition $+: V \times V \to V$,
    \item scalar multiplication $\cdot: \mathbb{F} \times V \to V$,
\end{itemize}
satisfying for all $u,v,w \in V$, $\alpha,\beta \in \mathbb{F}$:
\begin{enumerate}
    \item $u+v = v+u$ \hfill (commutativity)
    \item $(u+v)+w = u+(v+w)$ \hfill (associativity)
    \item There exists $0 \in V$ such that $v+0=v$ \hfill (additive identity)
    \item For each $v$, there exists $-v$ with $v+(-v)=0$ \hfill (additive inverse)
    \item $\alpha(u+v) = \alpha u + \alpha v$ \hfill (distributivity I)
    \item $(\alpha+\beta) v = \alpha v + \beta v$ \hfill (distributivity II)
    \item $(\alpha\beta)v = \alpha(\beta v)$ \hfill (associativity of scalar mult.)
    \item $1 \cdot v = v$ \hfill (unit property)
\end{enumerate}
\end{blue*}
\end{frame}

% --- Slide 2: Remark ---
\redheader
\begin{frame}{Abstract Linear Algebra: Remark}
\begin{red*}{Remark}
A vector space by itself has only the structure given by addition and scalar multiplication.  
\medskip

\begin{itemize}
    \item It does \emph{not} have a notion of \textbf{length}, \textbf{angle}, or \textbf{orthogonality}.
    \item These geometric notions are \textbf{induced} when we add an \textbf{inner product}, and from it, a norm.
\end{itemize}
\end{red*}
\end{frame}

% --- Slide 3: Inner Product ---
\blueheader
\begin{frame}{Abstract Linear Algebra: Inner Product}
\begin{blue*}{Definition}
An \textbf{inner product} on a real vector space $V$ is a function
\[
\langle \cdot, \cdot \rangle : V \times V \to \mathbb{R}
\]
such that for all $u,v,w \in V$, $\alpha \in \mathbb{R}$:
\begin{enumerate}
    \item $\langle u,v \rangle = \langle v,u \rangle$ \hfill (symmetry)
    \item $\langle u+v,w \rangle = \langle u,w \rangle + \langle v,w \rangle$ \hfill (linearity)
    \item $\langle \alpha u, v \rangle = \alpha \langle u,v \rangle$ \hfill (homogeneity)
    \item $\langle v,v \rangle \ge 0$, and $\langle v,v \rangle = 0 \iff v=0$ \hfill (positivity)
\end{enumerate}
\end{blue*}
\end{frame}

% --- Slide 4: Norm ---
\redheader
\begin{frame}{Abstract Linear Algebra: Norm}
\begin{red*}{Definition}
Given an inner product $\langle \cdot, \cdot \rangle$ on $V$, the induced \textbf{norm} is defined by
\[
\|v\| = \sqrt{\langle v, v \rangle}, \quad v \in V.
\]

This norm introduces the notions of:
\begin{itemize}
    \item \textbf{Length}: $\|v\|$
    \item \textbf{Angle}: $\cos \theta = \dfrac{\langle u, v \rangle}{\|u\|\|v\|}$
    \item \textbf{Orthogonality}: $u \perp v \iff \langle u, v \rangle = 0$
\end{itemize}
\end{red*}
\end{frame}

% --- Slide 5: Example ---
\greenheader
\begin{frame}{Abstract Linear Algebra: Example}
\begin{green*}{Example}
Let $V = \mathbb{R}^2$ with the standard inner product
\[
\langle u, v \rangle = u_1 v_1 + u_2 v_2.
\]

For $u = (3,4)$ and $v = (4,-3)$:
\[
\langle u, v \rangle = 3 \cdot 4 + 4 \cdot (-3) = 12 - 12 = 0.
\]

Thus $u \perp v$.  
Also $\|u\| = \sqrt{3^2+4^2}=5$.
\end{green*}
\end{frame}

% --- Slide 6: Exercise ---
\cyanheader
\begin{frame}{Abstract Linear Algebra: Exercise}
\begin{cyan*}{Exercise}
Let $u=(1,2,2)$ and $v=(2,0,1)$ in $\mathbb{R}^3$ with the standard inner product.
\begin{enumerate}
    \item Compute $\langle u,v \rangle$.
    \item Find $\|u\|$ and $\|v\|$.
    \item Determine the cosine of the angle between $u$ and $v$.
\end{enumerate}
\end{cyan*}
\end{frame}

% \include{Italian-Pronunciationa.tex}
% \section{Unit 1: French Pronunciation}

% --- Slide 1: Overview ---
\blueheader
\begin{frame}{French Pronunciation: Overview}
\begin{itemize}
  \item \textbf{Vowels}: oral vs nasal vowels.
  \item \textbf{Accents}: acute, grave, circumflex can change vowel quality.
  \item \textbf{Consonants}: many final consonants are silent.
  \item \textbf{Liaison}: linking a final consonant to the next word.
  \item \textbf{Rhythm}: syllable-timed, smoother than English.
\end{itemize}

\medskip
\infoCustomMsg{French spelling is less phonetic than Italian. Pronunciation rules are essential.}
\end{frame}

% --- Slide 2: Vowels ---
\redheader
\begin{frame}{Vowels with IPA}
\begin{itemize}
  \item Oral vowels: \textbf{a} (\textit{papa} \textipa{[papa]}), \textbf{e} close as in \textit{caf\'e} \textipa{[kafe]}, \textbf{i} (\textit{ici} \textipa{[isi]}), \textbf{o} (\textit{eau} \textipa{[o]}), \textbf{u} (\textit{lune} \textipa{[lyn]}).
  \item Nasal vowels: \textbf{an/en} \textipa{[A\~{}]} (\textit{enfant} \textipa{[A\~{}fA\~{}]}), \textbf{on} \textipa{[O\~{}]} (\textit{nom} \textipa{[nO\~{}]}), \textbf{in} \textipa{[E\~{}]} (\textit{vin} \textipa{[vE\~{}]}), \textbf{un} \textipa{[9\~{}]} (\textit{un} \textipa{[\~{}9]} or \textipa{[9\~{}]}).
  \item Accents matter: \textbf{\'e} \textipa{[e]} is different from \textbf{\`e} \textipa{[E]}.
\end{itemize}

\medskip
\infoCustomMsg[teal]{Keep vowels short and precise. Avoid English-style glides.}
\end{frame}

% --- Slide 3: Consonants ---
\blueheader
\begin{frame}{Consonants with IPA}
\begin{itemize}
  \item Many final consonants are silent: \textit{grand} \textipa{[gRA\~{}]}, \textit{petit} \textipa{[p@ti]}.
  \item French \textbf{r} is uvular \textipa{[R]} produced at the back of the throat.
  \item \textbf{h} is silent. Some words block liaison: \textit{le h\'eros} \textipa{[l@ eRo]}.
  \item Soft vs hard before \textit{e,i} similar to Italian: \textit{gar\c{c}on} \textipa{[gaRsO\~{}]}, \textit{gilet} \textipa{[ZilE]}.
\end{itemize}

\medskip
\infoCustomMsg[blue]{Avoid pronouncing final consonants unless rules require it.}
\end{frame}

% --- Slide 4: Liaison ---
\redheader
\begin{frame}{Liaison with IPA}
\begin{itemize}
  \item A normally silent final consonant is pronounced before a following vowel sound.
  \item \textbf{les amis} \textipa{[lez ami]}  $\to$ final \textit{s} sounds like \textipa{[z]}.
  \item \textbf{vous avez} \textipa{[vu zave]}  $\to$ linking \textit{s} becomes \textipa{[z]}.
  \item Not all liaisons are obligatory. Some are optional or forbidden.
\end{itemize}

\medskip
\infoCustomMsg[teal]{Mastering liaison makes your French sound natural.}
\end{frame}

% --- Slide 5: Rhythm and Stress ---
\blueheader
\begin{frame}{Rhythm and Stress}
\begin{itemize}
  \item French is syllable-timed. Syllables are similar in length.
  \item Stress falls on the last syllable of a phrase group, not each word.
  \item Intonation patterns carry much of the meaning.
\end{itemize}

\medskip
\infoCustomMsg{French rhythm feels smoother compared to English.}
\end{frame}

% --- Slide 6: Exercise ---
\cyanheader
\begin{frame}{Exercise}
Read aloud, mark stress, and practice liaison:
\begin{enumerate}
  \item \textbf{les amis} \textipa{[lez ami]}, \textbf{vous avez} \textipa{[vu zave]}
  \item \textbf{grand homme} \textipa{[gRA\~{} tOm]}, \textbf{petit enfant} \textipa{[p@ti tA\~{}fA\~{}]}
  \item \textbf{caf\'e} \textipa{[kafe]}, \textbf{p\`ere} \textipa{[pER]}, \textbf{lune} \textipa{[lyn]}, \textbf{vin} \textipa{[vE\~{}]}
\end{enumerate}

\medskip
\infoCustomMsg[cyan]{Focus on nasal vowels and smooth linking.}
\end{frame}

% --- Slide 7: Cheat-Sheet ---
\greenheader
\begin{frame}{Cheat-Sheet}
\begin{itemize}
  \item Oral vs nasal vowels: \textipa{[a,e,i,o,y]} vs \textipa{[A\~{},O\~{},E\~{},9\~{}]}.
  \item Accents change quality: \textbf{\'e} \textipa{[e]} vs \textbf{\`e} \textipa{[E]}.
  \item Final consonants mostly silent; French \textbf{r} is \textipa{[R]}.
  \item Liaison: link final consonant to next vowel sound \textipa{[lez ami]}.
  \item Rhythm: syllable-timed, phrase-final stress.
\end{itemize}
\end{frame}

% --- Slide 8: Motivation (beyond curriculum) ---
\redheader
\begin{frame}{Why it matters}
\begin{itemize}
  \item \textbf{International}: French is spoken on five continents.
  \item \textbf{Culture}: Films, songs, literature.
  \item \textbf{Careers}: Diplomacy, engineering, fashion, cuisine.
\end{itemize}

\infoCustomMsg{Not on the exam but key for real-world communication and culture.}
\end{frame}

% --- Slide 9: Listening Drill (Minimal Pairs) ---
\cyanheader
\begin{frame}{Listening Drill: Minimal Pairs}
\begin{itemize}
  \item \textbf{beau} \textipa{[bo]} vs \textbf{bon} \textipa{[bO\~{}]}  (oral vs nasal)
  \item \textbf{p\`ere} \textipa{[pER]} vs \textbf{pair} \textipa{[pER]}  (same sound, different meaning)
  \item \textbf{petit} \textipa{[p@ti]} vs \textbf{petite} \textipa{[p@tit]}  (silent vs pronounced final consonant)
  \item \textbf{les amis} \textipa{[lez ami]} vs \textbf{les hommes} \textipa{[lez Om]}  (liaison)
  \item \textbf{lune} \textipa{[lyn]} vs \textbf{loup} \textipa{[lu]}  (u \textipa{[y]} vs ou \textipa{[u]})
\end{itemize}

\medskip
\infoCustomMsg[cyan]{Listen, repeat, and exaggerate contrasts when practicing.}
\end{frame}


\end{document}

