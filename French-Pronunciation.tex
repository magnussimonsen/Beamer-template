\section{Unit 1: French Pronunciation}

% --- Slide 1: Overview ---
\blueheader
\begin{frame}{French Pronunciation: Overview}
\begin{itemize}
  \item \textbf{Vowels}: oral vs nasal vowels.
  \item \textbf{Accents}: acute, grave, circumflex can change vowel quality.
  \item \textbf{Consonants}: many final consonants are silent.
  \item \textbf{Liaison}: linking a final consonant to the next word.
  \item \textbf{Rhythm}: syllable-timed, smoother than English.
\end{itemize}

\medskip
\infoCustomMsg{French spelling is less phonetic than Italian. Pronunciation rules are essential.}
\end{frame}

% --- Slide 2: Vowels ---
\redheader
\begin{frame}{Vowels with IPA}
\begin{itemize}
  \item Oral vowels: \textbf{a} (\textit{papa} \textipa{[papa]}), \textbf{e} close as in \textit{caf\'e} \textipa{[kafe]}, \textbf{i} (\textit{ici} \textipa{[isi]}), \textbf{o} (\textit{eau} \textipa{[o]}), \textbf{u} (\textit{lune} \textipa{[lyn]}).
  \item Nasal vowels: \textbf{an/en} \textipa{[A\~{}]} (\textit{enfant} \textipa{[A\~{}fA\~{}]}), \textbf{on} \textipa{[O\~{}]} (\textit{nom} \textipa{[nO\~{}]}), \textbf{in} \textipa{[E\~{}]} (\textit{vin} \textipa{[vE\~{}]}), \textbf{un} \textipa{[9\~{}]} (\textit{un} \textipa{[\~{}9]} or \textipa{[9\~{}]}).
  \item Accents matter: \textbf{\'e} \textipa{[e]} is different from \textbf{\`e} \textipa{[E]}.
\end{itemize}

\medskip
\infoCustomMsg[teal]{Keep vowels short and precise. Avoid English-style glides.}
\end{frame}

% --- Slide 3: Consonants ---
\blueheader
\begin{frame}{Consonants with IPA}
\begin{itemize}
  \item Many final consonants are silent: \textit{grand} \textipa{[gRA\~{}]}, \textit{petit} \textipa{[p@ti]}.
  \item French \textbf{r} is uvular \textipa{[R]} produced at the back of the throat.
  \item \textbf{h} is silent. Some words block liaison: \textit{le h\'eros} \textipa{[l@ eRo]}.
  \item Soft vs hard before \textit{e,i} similar to Italian: \textit{gar\c{c}on} \textipa{[gaRsO\~{}]}, \textit{gilet} \textipa{[ZilE]}.
\end{itemize}

\medskip
\infoCustomMsg[blue]{Avoid pronouncing final consonants unless rules require it.}
\end{frame}

% --- Slide 4: Liaison ---
\redheader
\begin{frame}{Liaison with IPA}
\begin{itemize}
  \item A normally silent final consonant is pronounced before a following vowel sound.
  \item \textbf{les amis} \textipa{[lez ami]}  $\to$ final \textit{s} sounds like \textipa{[z]}.
  \item \textbf{vous avez} \textipa{[vu zave]}  $\to$ linking \textit{s} becomes \textipa{[z]}.
  \item Not all liaisons are obligatory. Some are optional or forbidden.
\end{itemize}

\medskip
\infoCustomMsg[teal]{Mastering liaison makes your French sound natural.}
\end{frame}

% --- Slide 5: Rhythm and Stress ---
\blueheader
\begin{frame}{Rhythm and Stress}
\begin{itemize}
  \item French is syllable-timed. Syllables are similar in length.
  \item Stress falls on the last syllable of a phrase group, not each word.
  \item Intonation patterns carry much of the meaning.
\end{itemize}

\medskip
\infoCustomMsg{French rhythm feels smoother compared to English.}
\end{frame}

% --- Slide 6: Exercise ---
\cyanheader
\begin{frame}{Exercise}
Read aloud, mark stress, and practice liaison:
\begin{enumerate}
  \item \textbf{les amis} \textipa{[lez ami]}, \textbf{vous avez} \textipa{[vu zave]}
  \item \textbf{grand homme} \textipa{[gRA\~{} tOm]}, \textbf{petit enfant} \textipa{[p@ti tA\~{}fA\~{}]}
  \item \textbf{caf\'e} \textipa{[kafe]}, \textbf{p\`ere} \textipa{[pER]}, \textbf{lune} \textipa{[lyn]}, \textbf{vin} \textipa{[vE\~{}]}
\end{enumerate}

\medskip
\infoCustomMsg[cyan]{Focus on nasal vowels and smooth linking.}
\end{frame}

% --- Slide 7: Cheat-Sheet ---
\greenheader
\begin{frame}{Cheat-Sheet}
\begin{itemize}
  \item Oral vs nasal vowels: \textipa{[a,e,i,o,y]} vs \textipa{[A\~{},O\~{},E\~{},9\~{}]}.
  \item Accents change quality: \textbf{\'e} \textipa{[e]} vs \textbf{\`e} \textipa{[E]}.
  \item Final consonants mostly silent; French \textbf{r} is \textipa{[R]}.
  \item Liaison: link final consonant to next vowel sound \textipa{[lez ami]}.
  \item Rhythm: syllable-timed, phrase-final stress.
\end{itemize}
\end{frame}

% --- Slide 8: Motivation (beyond curriculum) ---
\redheader
\begin{frame}{Why it matters}
\begin{itemize}
  \item \textbf{International}: French is spoken on five continents.
  \item \textbf{Culture}: Films, songs, literature.
  \item \textbf{Careers}: Diplomacy, engineering, fashion, cuisine.
\end{itemize}

\infoCustomMsg{Not on the exam but key for real-world communication and culture.}
\end{frame}

% --- Slide 9: Listening Drill (Minimal Pairs) ---
\cyanheader
\begin{frame}{Listening Drill: Minimal Pairs}
\begin{itemize}
  \item \textbf{beau} \textipa{[bo]} vs \textbf{bon} \textipa{[bO\~{}]}  (oral vs nasal)
  \item \textbf{p\`ere} \textipa{[pER]} vs \textbf{pair} \textipa{[pER]}  (same sound, different meaning)
  \item \textbf{petit} \textipa{[p@ti]} vs \textbf{petite} \textipa{[p@tit]}  (silent vs pronounced final consonant)
  \item \textbf{les amis} \textipa{[lez ami]} vs \textbf{les hommes} \textipa{[lez Om]}  (liaison)
  \item \textbf{lune} \textipa{[lyn]} vs \textbf{loup} \textipa{[lu]}  (u \textipa{[y]} vs ou \textipa{[u]})
\end{itemize}

\medskip
\infoCustomMsg[cyan]{Listen, repeat, and exaggerate contrasts when practicing.}
\end{frame}
